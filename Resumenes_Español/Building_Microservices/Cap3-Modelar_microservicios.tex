\documentclass[11pt,a4paper]{article}
\usepackage[utf8]{inputenc}
\usepackage[spanish]{babel}
\usepackage{amsmath}
\usepackage{amsfonts}
\usepackage{amssymb}
\usepackage{graphicx}
\author{Víctor Iranzo}
\title{Capítulo 3: Modelar microservicios}
\setlength{\parskip}{10pt}

\begin{document}
\maketitle

\section{Alta cohesión y bajo acoplamiento}

Estos dos términos son claves para implementar servicios de buena calidad:

\begin{itemize}

\item Alta cohesión: funcionalidades relacionadas deben agruparse juntas y separadas de otras no relacionadas. De esta manera, cuando se realice un cambio sólo se deberá modificar y desplegar un microservicio, haciendo el despliegue más seguro y coherente. Además, agrupar funcionalidades similares reducirá el número de llamadas potenciales a otros servicios.

\item Bajo acoplamiento: en los microservicios poco acoplados un cambio en uno de ellos no requiere cambiar ningún otro. Esta propiedad es clave ya que de otra manera no se puede desplegar de manera independiente un servicio sin cambiar otros que lo consuman. Un servicio debe conocer lo mínimo posible de otros con los que colabore. Un síntoma para detectar que dos servicios están acoplados es que estén en continua comunicación (el término en inglés es chatty communication).

\end{itemize}

\section{Contexto de un microservicio}

En el libro 'Domain-Driven Design', Eric Evans explica que el dominio de una solución está compuesta de múltiples contextos bien limitados. Cada contexto está formado por modelos que no necesitan ser compartidos con otros contextos a menos que se defina explícitamente una interfaz que lo emplee. La interfaz es el punto de entrada para que otros puedan comunicar con nuestro contexto, empleando los términos y entidades que en los modelos que aquí se definan.

Esta perspectiva puede trasladarse fácilmente al modelado de microservicios. Los contextos limitados de Evans que analicemos en nuestro sistema son firmes candidatos a transformarse en servicios. Así, los límites de un servicio quedan bien limitados porque  todas las entidades que pueda requerir se encuentran dentro de sus fronteras, garantizándose su alta cohesión y bajo acoplamiento.

No todos los modelos de un servicio son públicos. Por ejemplo, podemos definir una entidad en el modelo con muchos atributos pero solo hacer públicos al resto de servicios parte de ellos. También puede ocurrir que una misma entidad aparezca en modelos de distintos servicios. En este caso, los atributos de la entidad en cada uno de ellos podrían variar porque a cada uno les interese unas propiedades u otras.



\end{document}
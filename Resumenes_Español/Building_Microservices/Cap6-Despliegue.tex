\documentclass[11pt,a4paper]{article}
\usepackage[utf8]{inputenc}
\usepackage[spanish]{babel}
\usepackage{amsmath}
\usepackage{amsfonts}
\usepackage{amssymb}
\usepackage{graphicx}
\author{Víctor Iranzo}
\title{Capítulo 6: Despliegue}
\setlength{\parskip}{10pt}

\begin{document}
\maketitle

\section{Integración continua}

\subsection{¿Qué es la integración continua?}

\subsection{Relación entre microservicios y CI}

\subsection{Entrega continua}

\section{Artefactos}

\subsection{Artefactos específicos de la plataforma}

\subsection{Artefactos del sistema operativo}

\subsection{Imágenes propias}

\section{Entornos y configuración de servicios}

\subsection{Definición del entorno}

\subsection{Interfaz para el despliegue}

\section{Alojamiento de servicios}

\subsection{Múltiples servicios por host}

\subsection{Contenedores de aplicaciones}

\subsection{Un servicio por host}

\subsection{Plataforma como servicio (PaaS)}

\section{Automatización del despliegue}

\section{Tecnologías para el despliegue}

\subsection{Virtualización tradicional}

\subsection{Vagrant}

\subsection{Contenedores Linux}

\subsection{Docker}

\end{document}
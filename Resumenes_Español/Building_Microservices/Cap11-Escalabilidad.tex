\documentclass[11pt,a4paper]{article}
\usepackage[utf8]{inputenc}
\usepackage[spanish]{babel}
\usepackage{amsmath}
\usepackage{amsfonts}
\usepackage{amssymb}
\usepackage{graphicx}
\usepackage{float}

\author{Víctor Iranzo}
\title{Capítulo 11: Escalabilidad}
\setlength{\parskip}{10pt}

\begin{document}
\maketitle

\section{Introducción}

\section{Gestión de fallos en el sistema}

\subsection{Timeouts}

\subsection{Cortocircuitos}

\subsection{Bulkhead}

\subsection{Aislamiento}

\section{Idempotencia}

\section{Técnicas de escalabilidad}
%Incluir secciones Starting again y dividir el riesgo.

\subsection{Escalabilidad vertical}

\subsection{Dividir la carga de trabajo de un servicio}

\subsection{Balanceadores de carga}

\subsection{Sistema basado en trabajadores}

\section{Escalabilidad en bases de datos}
%Incluir sección de infraestructura de base de datos compartida

\subsection{Escalabilidad en la lectura}

\subsection{Escalabilidad en la escritura}

\subsection{CQRS}

\section{Caching}
%Incluir secciones de caching en la escritura, caching para resilencia y mantener sencillo.

\subsection{Tipos de caching}
%Incluir sección de HTTP Caching

\section{Autoescalado}

\begin{itemize}
\item Escalado reactivo:
\item Escalado predictivo:
\end{itemize}

\section{El teorema de CAP}
%Incluir sección de Sacrificio de Particiones tolerantes

\subsection{Sistemas AP}

\subsection{Sistemas CP}

\subsection{Balance entre sistemas AP y sistemas CP}

\section{Descubrimiento de servicios}
%Incluir sección no olvidarse de los humanos

\subsection{DNS}

\subsection{Herramientas para el descubrimiento de servicios}

\begin{itemize}
\item Zookeeper
\item Consul
\item Eureka
\end{itemize}

\section{Documentación de servicios}

\begin{itemize}
\item Swagger
\item HAL
\end{itemize}

\section{Descripción de un servicio}

\end{document}
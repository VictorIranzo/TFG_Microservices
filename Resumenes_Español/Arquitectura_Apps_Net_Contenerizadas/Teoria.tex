\documentclass[11pt,a4paper]{article}
\usepackage[utf8]{inputenc}
\usepackage[spanish]{babel}
\usepackage{amsmath}
\usepackage{amsfonts}
\usepackage{amssymb}
\usepackage{graphicx}
\usepackage{float}

\author{Víctor Iranzo}
\title{Microsoft .NET: Arquitectura para Aplicaciones .NET Contenerizadas}
\setlength{\parskip}{10pt}

\begin{document}
\maketitle

\section{Introducción}

\subsection{Terminología Docker}

\section{.NET Core vs .NET Framework para contenedores Docker}

\section{Principios de diseño de contenedores}

\subsection{Arquitectura orientada a servicios (SOA)}

\subsection{Arquitectura de microservicios}

\section{Contenerizando aplicaciones monolíticas}

\section{Datos de un microservicio}

\subsection{Soberanía de datos por microservicio}

\subsection{Datos y el patrón Bounded Context}

\subsection{Retos y soluciones para la gestión de datos distribuidos}

\subsection{Datos persistentes en aplicaciones Docker}

\section{Límites del modelo de dominio de un microservicio}

\section{Integración de microservicios}

\subsection{Patrón API Gateway}

\subsection{Tipos de comunicación}

\section{Evolución de los microservicios}

\section{Direccionabilidad de los microservicios}

\section{Interfaces de usuario}

\section{Resilencia y monitorización de los microservicios}

\subsection{Orquestadores}

\end{document}
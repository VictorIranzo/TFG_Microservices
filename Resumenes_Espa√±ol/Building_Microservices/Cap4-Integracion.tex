\documentclass[11pt,a4paper]{article}
\usepackage[utf8]{inputenc}
\usepackage[spanish]{babel}
\usepackage{amsmath}
\usepackage{amsfonts}
\usepackage{amssymb}
\usepackage{graphicx}
\author{Víctor Iranzo}
\title{Capítulo 4: Integración de microservicios}
\setlength{\parskip}{10pt}

\begin{document}
\maketitle

La integración de servicios es la parte más relevante en los sistemas basados en este concepto. Hacerlo correctamente nos asegurará su autonomía y su despliegue de manera independiente. Existen muchas tecnologías para la integración: SOAP (Simple Object Access Protocol), RPC (Remote Procedure Call) o REST (Representational State Transfer). De cualquiera de estas tecnologías esperamos las siguientes características:

\begin{itemize}

\item Evitar cambios en los consumidores: la tecnología escogida debe hacer que el número de cambios en un servicio que impliquen cambios en sus consumidores sean los menos posibles.

\item No imponer una tecnología específica: la tecnología empleada para la comunicación entre servicios no debe restringir la tecnología empleada en estos. Se debe mantener la heterogeneidad tecnológica de los servicios y el protocolo empleado para integrarlos debe poderse emplear en cuantas más tecnologías mejor.

\item Hacer simple el consumo de un servicio: los consumidores deberían de tener total libertad en la tecnología que emplean y consumir un servicio para ellos no debe ser complejo de implementar.

\item Ocultar detalles de la implementación: el consumidor de un servicio no debe conocer los detalles de como este está implementado internamente. Así, los interlocutores están desacoplados y se evitan cambios en el consumidor asociados al servicio.

\item Soportar operaciones más allá de las CRUD: las operaciones CRUD para crear, leer, actualizar y eliminar elementos están soportadas en la mayoría de tecnologías de integración. Sin embargo, un sistema requiere dar soporte a más procesos que se deben poder exponer en una interfaz de un servicio.

\end{itemize}

\section{Integración por base de datos}

\section{Integración REST y RPC}

\section{Patrones orquestador y coreógrafo}

\section{La ley de Postel y los lectores tolerantes}

\section{Integración en las interfaces de usuario}

\section{Integración con servicios de terceros}

\section{Integración con sistemas legados}

\end{document}
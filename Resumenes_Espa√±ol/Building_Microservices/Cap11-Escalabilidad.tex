\documentclass[11pt,a4paper]{article}
\usepackage[utf8]{inputenc}
\usepackage[spanish]{babel}
\usepackage{amsmath}
\usepackage{amsfonts}
\usepackage{amssymb}
\usepackage{graphicx}
\usepackage{float}

\author{Víctor Iranzo}
\title{Capítulo 11: Escalabilidad}
\setlength{\parskip}{10pt}

\begin{document}
\maketitle

\section{Introducción}

Podemos hacer todo lo posible para evitar que un sistema falle, pero llego un punto en la escala, la probabilidad de fallo es inevitable. Muchas organizaciones invierten mucho esfuerzo en evitar que un fallo se produzca, pero muy poco en mecanismos para recuperar el sistema una vez se ha producido.

Los mecanismos de escalabilidad de nuestro sistema han de ser conformes a sus necesidades: tener un sistema capaz de reaccionar automáticamente al incremento de carga o a un fallo es fantástico pero exagerado para ciertos servicios. También es necesario cuantos fallos pueden tolerar o cuánto tiempo puede estar el sistema caído para nuestros usuarios. Estos requisitos se pueden definir así:

\begin{itemize}

\item Latencia o tiempo de respuesta: cuánto tiempo puede una operación tardar. La carga que sufre un sistema puede influir en su tiempo de respuesta.

\item Disponibilidad: cuánto tiempo está disponible un servicio respecto del tiempo que se deseaba que estuviera en funcionamiento. ¿Puede estar el servicio caído o tiene que estar disponible 24/7?

\item Durabilidad de los datos: cuánto tiempo se deben almacenar ciertos datos y cuánta pérdida de datos se considera aceptable.

\item Resilencia: es la capacidad de un sistema para tolerar fallos y continuar trabajando. Un sistema que por culpa de un servicio caído deja de funcionar es menos resilente que un sistema que puede continuar ofreciendo el resto de sus funcionalidades.

\end{itemize}

Existen varios modos de fallo que se pueden dar cuando un sistema está caído. Responder de forma lenta es uno de los peores porque aumenta la latencia de todas las peticiones para responder que el servicio está caído. Esto puede producir fallos en cascada dentro de la cadena de invocaciones.

\section{Gestión de fallos en el sistema}

En el libro ``Antifragile" se explica como una organización como Netflix basada en la infraestructura de AWS prueba la tolerancia a fallos de sus servicios incitando al fallo de estos. Para ello, emplea un conjunto de programas que componen el ``ejército de simios". Chaos Monkey se encarga durante ciertas horas al día de apagar máquinas de forma aleatoria. Chaos Gorilla hace una función similar pero con los centros de datos. Latency Monkey simula redes de bajo rendimiento.

Todos estos fallos pueden darse en producción y esta librería es una buena manera para comprobar si se está preparado para tolerarlos.

\subsection{Timeouts}

\subsection{Cortocircuitos}

\subsection{Bulkhead}

\subsection{Aislamiento}

\section{Idempotencia}

\section{Técnicas de escalabilidad}
%Incluir secciones Starting again y dividir el riesgo.

\subsection{Escalabilidad vertical}

\subsection{Dividir la carga de trabajo de un servicio}

\subsection{Balanceadores de carga}

\subsection{Sistema basado en trabajadores}

\section{Escalabilidad en bases de datos}
%Incluir sección de infraestructura de base de datos compartida

\subsection{Escalabilidad en la lectura}

\subsection{Escalabilidad en la escritura}

\subsection{CQRS}

\section{Caching}
%Incluir secciones de caching en la escritura, caching para resilencia y mantener sencillo.

\subsection{Tipos de caching}
%Incluir sección de HTTP Caching

\section{Autoescalado}

\begin{itemize}
\item Escalado reactivo:
\item Escalado predictivo:
\end{itemize}

\section{El teorema de CAP}
%Incluir sección de Sacrificio de Particiones tolerantes

\subsection{Sistemas AP}

\subsection{Sistemas CP}

\subsection{Balance entre sistemas AP y sistemas CP}

\section{Descubrimiento de servicios}
%Incluir sección no olvidarse de los humanos

\subsection{DNS}

\subsection{Herramientas para el descubrimiento de servicios}

\begin{itemize}
\item Zookeeper
\item Consul
\item Eureka
\end{itemize}

\section{Documentación de servicios}

\begin{itemize}
\item Swagger
\item HAL
\end{itemize}

\section{Descripción de un servicio}

\end{document}